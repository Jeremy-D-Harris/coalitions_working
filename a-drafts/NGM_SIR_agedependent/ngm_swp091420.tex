\documentclass[12pt]{article}
\usepackage[top=1in,left=1in, right = 1in, footskip=1in]{geometry}

\usepackage{graphicx}
\usepackage{xspace}
%\usepackage{adjustbox}

\newcommand{\comment}{\showcomment}
%% \newcommand{\comment}{\nocomment}

\newcommand{\showcomment}[3]{\textcolor{#1}{\textbf{[#2: }\textsl{#3}\textbf{]}}}
\newcommand{\nocomment}[3]{}

\newcommand{\jd}[1]{\comment{cyan}{JD}{#1}}
\newcommand{\swp}[1]{\comment{magenta}{SWP}{#1}}
\newcommand{\bmb}[1]{\comment{blue}{BMB}{#1}}
\newcommand{\djde}[1]{\comment{red}{DJDE}{#1}}

\newcommand{\eref}[1]{(\ref{eq:#1})}
\newcommand{\fref}[1]{Fig.~\ref{fig:#1}}
\newcommand{\Fref}[1]{Fig.~\ref{fig:#1}}
\newcommand{\sref}[1]{Sec.~\ref{#1}}
\newcommand{\frange}[2]{Fig.~\ref{fig:#1}--\ref{fig:#2}}
\newcommand{\tref}[1]{Table~\ref{tab:#1}}
\newcommand{\tlab}[1]{\label{tab:#1}}
\newcommand{\seminar}{SE\mbox{$^m$}I\mbox{$^n$}R}

\usepackage{amsthm}
\usepackage{amsmath}
\usepackage{amssymb}
\usepackage{amsfonts}
\usepackage[utf8]{inputenc} % make sure fancy dashes etc. don't get dropped

\usepackage{lineno}
\linenumbers

\usepackage[pdfencoding=auto, psdextra]{hyperref}

\usepackage{natbib}
\bibliographystyle{chicago}
\date{\today}

\usepackage{xspace}
\newcommand*{\ie}{i.e.\@\xspace}

\usepackage{color}

\newcommand{\Rx}[1]{\ensuremath{{\mathcal R}_{#1}}\xspace} 
\newcommand{\Ro}{\Rx{0}}
\newcommand{\Rc}{\Rx{\mathrm{c}}}
\newcommand{\Rs}{\Rx{\mathrm{s}}}
\newcommand{\RR}{\ensuremath{{\mathcal R}}\xspace}
\newcommand{\Rhat}{\ensuremath{{\hat\RR}}}
\newcommand{\Rintrinsic}{\ensuremath{{\mathcal R}_{\textrm{\tiny intrinsic}}}\xspace}
\newcommand{\Rapprox}{\ensuremath{{\mathcal R}_{\textrm{\tiny approx}}}\xspace}
\newcommand{\Rnaive}{\ensuremath{{\mathcal R}_{\textrm{\tiny naive}}}\xspace}
\newcommand{\tsub}[2]{#1_{{\textrm{\tiny #2}}}}
\newcommand{\dd}[1]{\ensuremath{\, \mathrm{d}#1}}
\newcommand{\dtau}{\dd{\tau}}
\newcommand{\dx}{\dd{x}}
\newcommand{\dsigma}{\dd{\sigma}}

\newcommand{\psymp}{\ensuremath{p}} %% primary symptom time
\newcommand{\ssymp}{\ensuremath{s}} %% secondary symptom time
\newcommand{\pinf}{\ensuremath{\alpha_1}} %% primary infection time
\newcommand{\sinf}{\ensuremath{\alpha_2}} %% secondary infection time

\newcommand{\psize}{{\mathcal P}} %% primary cohort size
\newcommand{\ssize}{{\mathcal S}} %% secondary cohort size

\newcommand{\gtime}{\tau_{\rm g}} %% generation interval
\newcommand{\gdist}{g} %% generation-interval distribution
\newcommand{\idist}{\ell} %% incubation period distribution

\newcommand{\total}{{\mathcal T}} %% total number of serial intervals

\usepackage{lettrine}

\newcommand{\dropcapfont}{\fontfamily{lmss}\bfseries\fontsize{26pt}{28pt}\selectfont}
\newcommand{\dropcap}[1]{\lettrine[lines=2,lraise=0.05,findent=0.1em, nindent=0em]{{\dropcapfont{#1}}}{}}

\begin{document}

\begin{flushleft}{
	\Large
	\textbf\newline{
		Next generation matrix calculation
	}
}
\newline
\end{flushleft}

\section{Model}

Let $S_i$ represent the proportion of susceptible individuals in age group $i$ and $I_i$ represent the proportion of infected individuals in age group $i = 1, \dots, n$. Then, the model is given by:
\begin{align}
\frac{dI_{s,i}}{dt} &= \beta_a \sigma_{s, i} S_i \left(\sum_{j=1}^N C_{i, j} I_{a, j} \right) + \beta_s \sigma_{s, i} S_i \left(\sum_{j=1}^N C_{i, j} I_{s, j} \right) - \delta_s I_{s, i}\\
\frac{dI_{a,i}}{dt} &= \beta_a \sigma_{a, i} S_i \left(\sum_{j=1}^N C_{i, j} I_{a, j} \right) + \beta_s \sigma_{a, i} S_i \left(\sum_{j=1}^N C_{i, j} I_{s, j} \right) - \delta_a I_{s, i}
\end{align}

\section{NGM}

In this case, we have (using rough notations):
\begin{equation}
\mathcal F = \begin{pmatrix}
\beta_a \sigma_{s, i} S_i \left(\sum_{j=1}^N C_{i, j} I_{a, j} \right) + \beta_s \sigma_{s, i} S_i \left(\sum_{j=1}^N C_{i, j} I_{s, j} \right)\\
\beta_a \sigma_{a, i} S_i \left(\sum_{j=1}^N C_{i, j} I_{a, j} \right) + \beta_s \sigma_{a, i} S_i \left(\sum_{j=1}^N C_{i, j} I_{s, j} \right) 
\end{pmatrix}
\end{equation}
and
\begin{equation}
\mathcal V = \begin{pmatrix}
\gamma_s\\
\gamma_a
\end{pmatrix}
\end{equation}

Then, we can write $F$ and $V$ using a block matrix:
\begin{equation}
F = \begin{pmatrix}
F_{s,s,i,j} & F_{s,a,i,j}\\
F_{a,s,i,j} & F_{a,a,i,j}\\
\end{pmatrix}
\end{equation}
where $F_{x,y,i,j}$ represents transmission from infection type $y$ to $x$ from age group $j$ to $i$.
Then, we have
\begin{equation}
F_{x,y,i,j} = \beta_y \sigma_{x,i} C_{i,j}.
\end{equation}
Likewise,
\begin{equation}
V = \begin{pmatrix}
V_{s,s,i,j} & V_{s,a,i,j}\\
V_{a,s,i,j} & V_{a,a,i,j}\\
\end{pmatrix}
\end{equation}
In this case, $V_{s,a,i,j}$ and $V_{a,s,i,j}$ are zero matrices and $V_{s,s,i,j}$ and $V_{a,a,i,j}$ are diagonal matrices whose entries are $\delta_s$ and $\delta_a$, respectively.
Then, we have 
\begin{equation}
V^{-1} = \begin{pmatrix}
1/\delta_s & 0\\
0 & 1/\delta_a\\
\end{pmatrix}
\end{equation}
a $2n \times 2n$ diagonal matrix.

Then, the next generation matrix is given by:
\begin{equation}
FV^{-1} = \begin{pmatrix}
\mathcal R_s \sigma_{s,i} C_{i,j} & \mathcal R_a \sigma_{s,i} C_{i,j}\\
\mathcal R_s \sigma_{a,i} C_{i,j} & \mathcal R_a \sigma_{a,i} C_{i,j}\\
\end{pmatrix}
\end{equation}

If we want to calculate the exp growth rate, we take the largest eigenvalue of the matrix $F-V$. %>% 

\end{document}
